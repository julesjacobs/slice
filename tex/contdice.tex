\documentclass[nonacm,anonymous]{acmart}

\usepackage{listings}
\usepackage[scaled=0.84]{beramono}
\usepackage{mathpartir}
\usepackage{xspace}
\title{CDice: Discretizing Continuous Probabilistic Programs for BDD-based Inference}

\lstdefinelanguage{CDice}{
    morekeywords={let, in, if, then, else},
    morekeywords=[2]{uniform, discrete},
    morekeywords=[3]{bool, float},
    basicstyle=\ttfamily,
    keywordstyle=\bfseries,
    commentstyle=\itshape,
    stringstyle=\ttfamily,
}

% Macros for language keywords to be used in math mode
\newcommand{\letkw}{\text{\ttfamily\bfseries let}}
\newcommand{\inkw}{\text{\ttfamily\bfseries in}}
\newcommand{\ifkw}{\text{\ttfamily\bfseries if}}
\newcommand{\thenkw}{\text{\ttfamily\bfseries then}}
\newcommand{\elsekw}{\text{\ttfamily\bfseries else}}
\newcommand{\uniform}{\text{\ttfamily\bfseries uniform}}
\newcommand{\discrete}{\text{\ttfamily\bfseries discrete}}
\newcommand{\bool}{\text{\ttfamily\bfseries bool}}
\newcommand{\float}{\text{\ttfamily\bfseries float}}

% Macros for language names, use small caps
\newcommand{\CDice}{\text{\scshape CDice}\xspace}
\newcommand{\DDice}{\text{\scshape DDice}\xspace}

% Set this as the default language for listings
\lstset{language=CDice}

\begin{document}

\maketitle


\section{Introduction}

\begin{lstlisting}[language=CDice]
let x = uniform(0, 1) in
let y = uniform(0, 2) in
if x < 0.5 then x < 0.1 else y < 0.1
\end{lstlisting}

\begin{lstlisting}[language=CDice]
let x = discrete[0.1; 0.4; 0.5] in
let y = discrete[0.05; 0.95] in
if x <= 1 then x <= 0 else y <= 0
\end{lstlisting}

\section{Language}

The syntax of the continuous language \CDice is:
%
\begin{align*}
e ::= &\; x                               & \text{variable} \\
    | &\; \letkw \; x = e_1\; \inkw \; e_2  & \text{let binding} \\
    | &\; \uniform(c_1, c_2)      & \text{uniform distribution} \\
    | &\; e < c                           & \text{threshold test} \\
    | &\; \ifkw \; e_1\; \thenkw \; e_2\; \elsekw \; e_3 & \text{conditional}
\end{align*}

\noindent where $x$ ranges over variable names and $c$ ranges over floating point constants.

The syntax of the discrete language \DDice is similar, but replaces uniform distributions with discrete distributions over a fixed set of points:
%
\begin{align*}
e ::= &\; x                               & \text{variable} \\
    | &\; \letkw \; x = e_1\; \inkw \; e_2  & \text{let binding} \\
    | &\; \discrete[p_0; \ldots; p_{n}]      & \text{discrete distribution} \\
    | &\; e \leq c                           & \text{threshold test} \\
    | &\; \ifkw \; e_1\; \thenkw \; e_2\; \elsekw \; e_3 & \text{conditional}
\end{align*}

\noindent Note that threshold tests use $\leq$ in DDice rather than $<$ as in CDice.
The discrete distribution $\discrete[p_0; \ldots; p_{n}]$ is a distribution over natural numbers $\{0, \ldots, n\}$, with $p_i$ being the probability of observing $i$.

\section{Type System}

We introduce a type system that analyzes the threshold points of each floating point expression. We have two types:
\begin{itemize}
    \item \bool: the expression is a boolean value (true or false)
    \item \float$\langle c_0, \ldots, c_n \rangle$: the expression is a floating point value that can only be compared with threshold tests $e \leq c_i$ for $i \in \{0, \ldots, n\}$.
\end{itemize}

The typing rules are as follows:

\begin{mathpar}
    \inferrule[\textsc{Var}]
    {\ }
    {\Gamma, x: \tau \vdash x : \tau}

    \inferrule[\textsc{Let}]
    {\Gamma, x: \tau \vdash e_1 : \tau' \\
     \Gamma \vdash e_2 : \tau}
    {\Gamma \vdash \letkw \; x = e_1 \; \inkw \; e_2 : \tau}

    \inferrule[\textsc{If}]
    {\Gamma \vdash e_1 : \bool \\
     \Gamma \vdash e_2 : \tau \\
     \Gamma \vdash e_3 : \tau}
    {\Gamma \vdash \ifkw \; e_1 \; \thenkw \; e_2 \; \elsekw \; e_3 : \tau}

    \inferrule[\textsc{Uniform}]
    {\ }
    {\Gamma \vdash \uniform(c_1, c_2) : \float\langle c'_0, \ldots, c'_n \rangle}

    \inferrule[\textsc{Threshold}]
    {\Gamma \vdash e : \float\langle c_0, \ldots, c_n \rangle}
    {\Gamma \vdash e \leq c_i : \bool}
\end{mathpar}

Note that the typing rule for $\uniform(c_1, c_2)$ allows \emph{any} threshold points $c'_0, \ldots, c'_n$, unrelated to the original $c_1, c_2$.

\section{Inference}

\section{Discretization}

\end{document}
