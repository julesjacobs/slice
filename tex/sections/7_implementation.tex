\section{Implementation}\label{sec:implementation}

We have implemented \Slice{} as an OCaml-based compiler that transforms continuous probabilistic programs into discrete ones, which can then be analyzed using discrete exact inference engines, leveraging their benefits. At a high-level, our implementation consists of two main components: the \Slice{} compiler (\texttt{slice}) and an off-the-shelf discrete inference engine. Between these two components lies a conversion module that transforms the discretized program specfied in our syntax to a syntax compatible with a discrete inference backend, such as Dice and Roulette which we currently support. Figure~\ref{fig:architecture} shows the overall architecture of the \Slice{} system.

\begin{figure}[h]
\centering
\begin{tikzpicture}[
    node distance=1.5cm,
    component/.style={rectangle, draw, thick, minimum width=3cm, minimum height=1cm, align=center},
    data/.style={rectangle, draw, dashed, minimum width=2.5cm, minimum height=0.8cm, align=center},
    arrow/.style={->, thick},
    bigbox/.style={draw, thick, rounded corners, inner sep=0.3cm}
]
    % Input
    \node[data] (input) {\texttt{.cdice} file\\(continuous program)};
    
    % Slice compiler box
    \node[bigbox, below=0.5cm of input, minimum width=10cm, minimum height=7cm] (slicebox) {};
    \node[above] at (slicebox.north) {\textbf{\Slice{} Compiler (cdice)}};
    
    % Components inside Slice
    \node[component, below=1cm of input] (parser) {Parser\\(\texttt{parser.mly}, \texttt{lexer.mll})};
    \node[component, below of=parser] (typeinf) {Type Inference\\(\texttt{inference.ml})};
    \node[component, below of=typeinf] (disc) {Discretization\\(\texttt{discretization.ml})};
    \node[component, below of=disc] (todice) {Code Generation\\(\texttt{to\_dice.ml})};
    
    % Intermediate data structures
    \node[data, right=0.5cm of parser] (ast) {AST\\(\texttt{expr})};
    \node[data, right=0.5cm of typeinf] (tast) {Typed AST\\(\texttt{texpr})};
    \node[data, right=0.5cm of disc] (dast) {Discretized\\AST};
    
    % Bags module
    \node[component, left=0.5cm of typeinf, minimum width=2cm] (bags) {Bags\\(\texttt{bags.ml})};
    
    % Output from Slice
    \node[data, below=1cm of slicebox] (dice) {\texttt{.dice} file\\(discrete program)};
    
    % Dice engine
    \node[component, below=0.5cm of dice, minimum width=10cm] (diceengine) {Dice Inference Engine\\BDD compilation + Weighted Model Counting};
    
    % Final output
    \node[data, below=0.5cm of diceengine] (output) {Probability\\Distribution};
    
    % Arrows
    \draw[arrow] (input) -- (parser);
    \draw[arrow] (parser) -- (typeinf);
    \draw[arrow] (typeinf) -- (disc);
    \draw[arrow] (disc) -- (todice);
    \draw[arrow] (todice) -- (dice);
    \draw[arrow] (dice) -- (diceengine);
    \draw[arrow] (diceengine) -- (output);
    
    % Side arrows for data structures
    \draw[arrow, dashed] (parser) -- (ast);
    \draw[arrow, dashed] (typeinf) -- (tast);
    \draw[arrow, dashed] (disc) -- (dast);
    \draw[arrow, dashed] (bags) -- (typeinf);
\end{tikzpicture}
\caption{Architecture of the \Slice{} system. The \Slice{} compiler transforms continuous probabilistic programs into discrete ones through type-directed discretization. The resulting discrete program is then processed by the Dice inference engine to compute exact probability distributions.}
\label{fig:architecture}
\end{figure}

\subsection{Implementation Details}

\paragraph{Parser and Lexer} The frontend uses OCaml's Menhir parser generator to parse \texttt{.slice} source files. The parser (\texttt{parser.mly}) defines the concrete syntax and produces an abstract syntax tree (AST) represented by the \texttt{expr} type. The lexer (\texttt{lexer.mll}) handles tokenization, including special tokens for finite type comparisons (e.g., \texttt{<\#n}).

\paragraph{Type Inference} The type inference module (\texttt{inference.ml}) implements the two-bag type system described in Section~\ref{sec:type-inference}. It uses a constraint-based approach with unification to infer types while collecting comparison bounds. The implementation features:
\begin{itemize}
    \item A union-find data structure for managing bags efficiently
    \item Listener patterns for propagating constraints between related expressions
    \item Subtyping rules that ensure proper information flow
\end{itemize}

\paragraph{Bags Module} The \texttt{bags.ml} module provides the lattice-based data structures for tracking bounds and values. It implements:
\begin{itemize}
    \item Generic bag operations (union, equality checking)
    \item Specialized \texttt{BoundBag} for comparison bounds ($<c$, $\leq c$)
    \item \texttt{FloatBag} for tracking concrete float values
    \item Efficient representation using either finite sets or $\top$
\end{itemize}

\paragraph{Discretization} The discretization module (\texttt{discretization.ml}) transforms typed continuous programs into discrete ones. For each continuous distribution, it:
\begin{enumerate}
    \item Extracts comparison thresholds from the bound bag
    \item Computes interval probabilities using distribution CDFs
    \item Generates equivalent discrete distributions
    \item Transforms comparisons to operate on interval indices
\end{enumerate}

\paragraph{Code Generation} The \texttt{to\_dice.ml} and \texttt{to\_roulette.ml} module generates Dice and Roulette source code from the discretized AST, respectively. It handles the translation of \Slice{} constructs to their Dice and Roulette equivalents, including special handling for finite types that become integers in Dice.

\paragraph{Runtime and Evaluation} The implementation includes:
\begin{itemize}
    \item An interpreter (\texttt{interp.ml}) for sampling-based execution
    \item Statistical testing to verify discretization correctness
    \item Support for multiple output formats (Dice, SPPL)
\end{itemize}

\subsection{Integration with Backends}
We currently allow \Slice{} to be compatible to two different discrete inference engines: Dice provides exact inference for discrete probabilistic programs using binary decision diagrams (BDDs) and weighted model counting, and Roulette provides exact inference for discrete programs using an SMT solver with the aim of providing more expressivity. To work seamlessly with Dice, \Slice{} successfully maps 1) discrete distributions to Dice's \texttt{discrete} construct, 2) finite types to bounded integers of the form \texttt{int(size, value)}, and 3) observations to Dice's conditioning primitives. To work seamlessly with Roulette, \Slice{} successfully maps 1) discrete distributions to a chain of if-else statements with \texttt{flip}s, as Roulette currently does not support such a \texttt{discrete} construct, and 2) data structures, higher order functions, recursion, and mutable state as was previously unsupported by Dice to their Roulette counterparts. Finally, tying everything together is the \texttt{slice.py} script that automates the complete workflow, piping the output of \Slice{} directly to Dice or Roulette for inference.

% Listings setup
\lstset{
  basicstyle=\scriptsize\ttfamily,
  breaklines=true,
  columns=fullflexible,
  frame=none,
  backgroundcolor=\color{gray!5},
  showstringspaces=false
}

% Saveboxes for code blocks
\newsavebox{\boxone}
\begin{lrbox}{\boxone}
\begin{minipage}{5.2cm}
\begin{lstlisting}
let g = gaussian(0, 1) in
let check = g < 0.1 in
let x = if check then
    uniform(0, 0.2)
  else
    uniform(0.8, 1.0) in
x < 0.15
\end{lstlisting}
\end{minipage}
\end{lrbox}

\newsavebox{\boxtwo}
\begin{lrbox}{\boxtwo}
\begin{minipage}{5.2cm}
\begin{lstlisting}
let g = discrete(0.539828: 0#2, 0.460172: 1#2) in
let check = g <#2 1#2 in
  let x = if check then
        discrete(0.75: 0#2, 0.25: 1#2)
      else
        discrete(0: 0#2, 1: 1#2) in
    x <#2 1#2
\end{lstlisting}
\end{minipage}
\end{lrbox}

\newsavebox{\boxthree}
\begin{lrbox}{\boxthree}
\begin{minipage}{5.2cm}
\begin{lstlisting}
let g = discrete(0.539828, 0.460172) in
let check = g < int(1,1) in
  let x = if check then
        discrete(0.75, 0.25)
      else
        discrete(0., 1.) in
    x < int(1,1)
\end{lstlisting}
\end{minipage}
\end{lrbox}

\newsavebox{\boxfour}
\begin{lrbox}{\boxfour}
\begin{minipage}{5.2cm}
\begin{lstlisting}
(define g (if (flip 0.539828) 0 1))
(define check (< g 1))
(define x (if check 
               (if (flip 0.75) 0 1) 
               (if (flip 0.0) 0 1)))
(< x 1)
\end{lstlisting}
\end{minipage}
\end{lrbox}


\begin{center}
\begin{tikzpicture}[
    box/.style={
    draw,
    thick,
    align=left,
    inner sep=4pt,
    minimum width=5.4cm,
    minimum height=2.5cm
    },
    arrow/.style={
    -{Latex[length=2.5mm]},
    thick
    }
]

    % Top row
    \node[box] (box1) {\usebox{\boxone}};
    \node[box, right=1.6cm of box1] (box2) {\usebox{\boxtwo}};

    % Bottom row: tightly packed under box2
    \coordinate (belowcenter) at ($(box2.south) + (0,-2.2cm)$);
    \node[box, anchor=north east] (box3) at ($(belowcenter) + (-2.0cm, 0)$) {\usebox{\boxthree}};
    \node[box, anchor=north west] (box4) at ($(belowcenter) + (-1.0cm, 0)$) {\usebox{\boxfour}};

    % Arrows
    \draw[arrow] (box1) -- (box2);
    \draw[arrow] (box2.south) -- (box3.north);
    \draw[arrow] (box2.south) -- (box4.north);

\end{tikzpicture}
\end{center}

\subsection{Engineering Considerations}

The implementation prioritizes correctness and clarity:
\begin{itemize}
    \item Extensive use of OCaml's type system for safety
    \item Modular design allowing independent testing of components
    \item Clear separation between syntax-directed and semantic phases
    \item Comprehensive test suite including adversarial examples
\end{itemize}

The \Slice{} compiler consists of approximately 3,400 lines of OCaml code (including the lexer, parser, type inference engine, discretization engine, and code generator), demonstrating that powerful program transformations can be implemented concisely.

The system is open-source and available at \url{https://github.com/[repository-url]}, along with benchmarks and documentation.


