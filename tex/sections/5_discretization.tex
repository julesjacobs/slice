\section{Discretization}\label{sec:discretization}

After type inference, each float-typed expression carries refinement information in its type $\float[B; V]$. The discretization transformation uses this information to convert continuous probabilistic programs into discrete ones that can be analyzed exactly.

\subsection{Overview}

The discretization process transforms continuous distributions into discrete ones based on the comparison points collected during type inference. The key insight is that if a program only compares a continuous value against certain thresholds, we can partition the continuous range into intervals separated by these thresholds and track only which interval contains the value.

For each expression with type $\float[B; V]$:
\begin{itemize}
    \item If $B = \top$ (unbounded), the expression remains continuous
    \item If $B = \{b_1, \ldots, b_n\}$ (finite bounds), we create $n+1$ intervals
\end{itemize}

\subsection{Discretization Rules}

The transformation $\texttt{discretize}(e)$ is defined recursively on the structure of typed expressions:

\paragraph{Float Constants} For a constant $c$ with type $\float[B; V]$:
\begin{itemize}
    \item If $B = \top$: remains as constant $c$
    \item If $B = \{b_1, \ldots, b_n\}$: becomes $\finconst{k}{n+1}$ where $k$ is the index of the interval containing $c$
\end{itemize}

\paragraph{Continuous Distributions} For a distribution $d$ with type $\float[B; V]$ where $B = \{b_1, \ldots, b_n\}$:
\begin{enumerate}
    \item Extract numeric thresholds $c_1 < \cdots < c_n$ from bounds in $B$
    \item Let $\text{CDF}_d(b_i)$ be the measure of the distribution $d$ on the interval $(-\infty, c_i)$ if $b_i =\ <\!\!c_i$ and on the interval $(-\infty, c_i]$ if $b_i =\ \leq\!\!c_i$. Take $\text{CDF}_d(b_{0}) = 0$ and $\text{CDF}_d(b_{n+1}) = 1$.
    \item Calculate probability mass for each interval using the CDF:
    \[ p_i = \text{CDF}_d(b_i) - \text{CDF}_d(b_{i-1}) \]
    \item Generate $\discrete(p_1, \ldots, p_{n+1})$ returning interval indices.
\end{enumerate}

If $B = \top$, the distribution remains continuous.

\paragraph{Comparisons} For $e_1 < e_2$ where both have type $\float[B; V]$:
\begin{itemize}
    \item If $B = \top$: comparison remains as $\texttt{discretize}(e_1) < \texttt{discretize}(e_2)$
    \item If $B = \{b_1, \ldots, b_n\}$: becomes $\texttt{discretize}(e_1) \finlt{n+1} \texttt{discretize}(e_2)$ (or, $\leq$ becomes $\finleq{n+1}$)
\end{itemize}

The finite comparison $\finlt{n+1}$ operates on interval indices.

\paragraph{Structural Constructs} All other constructs are discretized recursively:
\begin{itemize}
    \item Variables pass through unchanged, though their type may change
    \item Let bindings, conditionals, pairs, etc., recursively discretize subexpressions
\end{itemize}

\paragraph{Effect of discretization on types} The types change as follows:
\begin{itemize}
    \item $\texttt{discretize}(\float[B; V]) = \fin(n+1)$ if $B = \{b_1, \ldots, b_n\}$
    \item $\texttt{discretize}(\float[B; V]) = \float$ if $B = \top$
    \item $\texttt{discretize}(A \to B) = \texttt{discretize}(A) \to \texttt{discretize}(B)$
    \item $\texttt{discretize}(A \times B) = \texttt{discretize}(A) \times \texttt{discretize}(B)$
    \item $\texttt{discretize}(\texttt{list}(A)) = \texttt{list}(\texttt{discretize}(A))$
    \item $\texttt{discretize}(\texttt{ref}(A)) = \texttt{ref}(\texttt{discretize}(A))$
    \item Other types remain unchanged.
\end{itemize}


\subsection{Example}

Consider the following program:
\begin{lstlisting}
let x = uniform(0.0, 1.0) in
if x < 0.5 then 
  if x < 0.7 then 1 else 2
else 3
\end{lstlisting}

After type inference, \texttt{x} has type $\float[\{<0.5, <0.7\}; \top]$. Discretization:
\begin{enumerate}
    \item Extracts thresholds: $\{0.5, 0.7\}$
    \item Creates intervals: $I_0 = (-\infty, 0.5)$, $I_1 = [0.5, 0.7)$, $I_2 = [0.7, \infty)$
    \item Calculates probabilities: $p_0 = 0.5$, $p_1 = 0.2$, $p_2 = 0.3$
    \item Generates: \texttt{let x = discrete(0.5, 0.2, 0.3) in ...}
    \item Transforms comparisons: \texttt{x < 0.5} becomes \texttt{x}$\finlt{3}$\texttt{1}, \texttt{x < 0.7} becomes \texttt{x}$\finlt{3}$\texttt{2}
\end{enumerate}

The discretized program preserves the original semantics while enabling exact inference.

\subsection{Variable Distribution Parameters}

A key challenge arises when distribution parameters are not constant.
For variable distribution parameters, we use the value bag $V$ in $\float[B; V]$ to reconstruct the appropriate distribution to compute the CDF on.
Due to the typing rules, each interval induced by $B$ can contain at most one value from $V$. This means that we can uniquely reconstruct the original floating point number from the interval index, by a series of conditionals.
The distribution can then be discretized as if its parameters were constant in each branch of the conditionals.

Consider the following program:
\begin{lstlisting}
let scale = discrete(0.5, 0.5) in
let x = if scale = 0 then exponential(1.0) 
        else exponential(2.0) in
x < 3.0
\end{lstlisting}

During type inference, the parameter to \texttt{exponential} has type $\float[B_{\text{param}}; \{1.0, 2.0\}]$. The implementation generates runtime dispatch code that:
\begin{enumerate}
    \item Evaluates the parameter expression
    \item Matches against possible constant values from the value bag
    \item Applies appropriate discretization for each case
\end{enumerate}

This ensures correct discretization even when distribution parameters depend on runtime values.


