\section{Introduction}\label{sec:intro}
Probabilistic programming languages (PPLs) have emerged as a powerful tool for building complex statistical models and reasoning about uncertainty~\cite{Moy2025Roulette,Holtzen2020Dice,DeRaedt2007ProbLog,Saad2021SPPL,Carpenter2017Stan,Salvatier2016PyMC3,Bingham2019Pyro,Dillon2017TFP,Tran2016Edward,Tolpin2016Anglican,Goodman2014WebPPL,Pfeffer2009Figaro,Minka2018InferNET,Ge2018Turing,CusumanoTowner2019Gen,Tehrani2020BeanMachine,Goodman2008Church}. A key challenge in PPLs is inference: computing the probability of an event given a model. The tractability of inference often depends on the kinds of random variables used. For programs with only discrete random variables, powerful exact inference engines can compute probabilities precisely. However, many real-world models require continuous random variables to represent quantities like time, distance, or sensor readings. For these programs, exact inference is often intractable, forcing practitioners to rely on approximate methods like Monte Carlo simulation or variational inference. 

Existing discrete PPLs like \Dice~\cite{Holtzen2020Dice} and Roulette~\cite{Moy2025Roulette} offer powerful, exact inference but cannot handle continuous distributions. At the other end of the spectrum, popular general-purpose PPLs like Stan~\cite{Carpenter2017Stan} and Pyro~\cite{Bingham2019Pyro} excel at approximate inference for continuous models but sacrifice exactness. Symbolic inference systems like SPPL~\cite{Saad2021SPPL} can reason exactly about certain mixed discrete-continuous models, but for a restricted class of programs.

Another related line of work is on probabilistic program abstraction.
For example, \cite{Holtzen2018Abstraction} use abstract interpretation to translate a hybrid discrete-continuous program into a finite Boolean program.
Their approach is distributionally sound, but it relies on the user to provide a set of predicates to abstract the program with.
In contrast, our work automatically infers discretization points from the program text itself.

This paper introduces \Slice{}, which occupies a unique position in the PPL design space: it automatically and soundly translates a broad class of programs with continuous variables into equivalent discrete programs. This automated, sound discretization unlocks the ability to use powerful discrete exact inference engines on models that were previously out of their reach. The key insight is to analyze how continuous variables are used within the program. Specifically, \Slice{} uses a type system to track all the constant thresholds against which continuous variables are compared. These comparison points are then used to intelligently discretize the continuous distributions into a finite number of buckets. When the resulting program is purely discrete it can then be solved by an off-the-shelf exact inference engine such as Dice~\cite{Holtzen2020Dice}. This type-directed discretization allows developers to model with continuous distributions while still benefiting from the speed and precision of exact discrete inference.

For instance, consider a simple program that checks if a uniform random variable is less than 0.5:
\begin{lstlisting}[aboveskip=1em,belowskip=1em]
    uniform(0, 1) < 0.5
\end{lstlisting}
\noindent \Slice{}'s type system determines that 0.5 is the only relevant "cut point." It automatically transforms the program into an equivalent discrete version:
\begin{lstlisting}[aboveskip=1em,belowskip=1em]
    discrete(0.5, 0.5) < 1
\end{lstlisting}
\noindent The continuous uniform distribution is replaced by a discrete one with two equally likely outcomes (0 and 1), and the comparison is transformed accordingly.

The true power of this approach, however, lies in its ability to handle programs where these relationships are not immediately obvious. A single continuous variable might be compared against multiple different constants across separate parts of the program. Further, these constraints are propagated through complex control flow. For example, a value drawn from a continuous distribution might be returned from one branch of a conditional, passed through a higher-order function, or stored in a data structure before it is eventually compared to a constant. \Slice{}'s type inference systematically tracks these dependencies, ensuring that all relevant cut points are associated with the original distribution, regardless of how the value flows through the program. This robust, whole-program analysis is the key to handling expressive models, as we demonstrate in \Cref{sec:examples}.

The \Slice{} language itself is a statically-typed functional language that supports a rich set of features, including higher-order functions, recursion, pairs, lists, and even mutable references. Our transformation handles all of these features, allowing programmers to write expressive, high-level models that can still be analyzed with exact discrete inference.

\paragraph{Contributions.}
We present and evaluate our framework for type-directed discretization via the following contributions:

\begin{description}
    \item[Type System (\Cref{sec:language})] We introduce a novel type system that identifies which expressions are discretizable by analyzing how they are compared against constant values throughout the program.

    \item[Type Inference (\Cref{sec:type-inference})] We develop a type inference algorithm that automatically collects the set of constant thresholds needed for discretization for each expression.

    \item[Discretization (\Cref{sec:discretization})] We present a type-directed program transformation that soundly converts continuous distributions and their comparisons into discrete counterparts.

    \item[Soundness (\Cref{sec:soundness})] We prove that our discretization method is sound, establishing that the discretized program preserves the semantics of the original program. Our proof employs logical relations, a well-established technique for reasoning about program equivalences in probabilistic languages~\cite{Bizjak2015Step,Wand2018Contextual}.

    \item[Implementation (\Cref{sec:implementation})] We provide an open-source implementation of \Slice{} that compiles programs for the \Dice{} exact inference engine. Our implementation also includes a direct interpreter for partially discretized programs.

    \item[Evaluation (\Cref{sec:evaluation})] We conduct an empirical evaluation on a range of benchmarks, demonstrating that our approach is significantly faster than prior symbolic techniques on hybrid programs while maintaining correctness.
\end{description}

Finally, we discuss related and future work (\Cref{sec:related}).

