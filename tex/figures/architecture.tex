\begin{figure}[h]
\centering
\begin{tikzpicture}[
    node distance=1.2cm and 2cm,
    data/.style={rectangle, draw, dashed, minimum width=2.5cm, minimum height=0.9cm, align=center, font=\small},
    arrow/.style={->, thick},
    interp/.style={->, thick},
    operation/.style={font=\footnotesize, midway, above},
    operationside/.style={font=\footnotesize, midway, right},
    operationleft/.style={font=\footnotesize, midway, left},
    bigbox/.style={draw, thick, rounded corners, inner sep=0.3cm},
    impl/.style={font=\tiny, text=gray}
]
    % Slice compiler pipeline (horizontal inside the box)
    \node[data] (ast) {AST\\{\tiny\texttt{expr}}};
    \node[data, right=of ast] (tast) {Typed AST\\{\tiny\texttt{texpr}}};
    \node[data, right=of tast] (dast) {Discretized\\{\tiny AST}};
    
    % Input above AST
    \node[data, above=1.5cm of ast] (input) {\texttt{.slice} file\\{\footnotesize continuous program}};
    
    % Slice compiler box
    \node[bigbox, fit=(ast)(tast)(dast), inner sep=0.5cm, label=above:{\textbf{\Slice{} Compiler}}] (slicebox) {};
    
    % ast.ml in top right corner of compiler box
    \node[impl] at ($(slicebox.south) + (0.0,0.2)$) {ast.ml, utils.ml, pretty.ml};
    
    % Output below dast
    \node[data, below=1.5cm of dast] (dice) {\texttt{.dice} file\\{\footnotesize discrete program}};
    
    % Dice engine result to the left of dice output
    \node[data, below=1.5cm of tast] (output) {Probability\\Distribution};
    
    % Arrows with operations
    % Input to compiler
    \draw[arrow] (input) -- node[operationside, yshift=5mm, xshift=1mm] {Parse} (ast);
    \node[impl, align=center] at ($(input)!0.5!(ast) + (0.6,0.1)$) {parser.mly\\lexer.mll};
    
    % Inside compiler
    \draw[arrow] (ast) -- node[operation] {Type Infer} (tast);
    \node[impl, align=center] at ($(ast)!0.5!(tast) + (0,-0.3)$) {inference.ml\\bags.ml(i)};


    \draw[arrow] (tast) -- node[operation] {Discretize} (dast);
    \node[impl, align=center] at ($(tast)!0.5!(dast) + (0,-0.3)$) {discretization.ml\\distributions.ml};
    
    % Compiler to output
    \draw[arrow] (dast) -- node[operationside, yshift=-1mm] {Convert} (dice);
    \node[impl, align=center] at ($(dast)!0.5!(dice) + (0.6,-0.4)$) {to\_dice.ml};
    
    % Dice engine - horizontal arrow from dice to output
    \draw[arrow] (dice) -- node[operation, align=center] {Dice Engine} (output);
    \node[impl, align=center] at ($(dice)!0.5!(output) + (0,-0.2)$) {BDD + WMC};
    
    % Interpreter arrows (dashed blue)
    \draw[interp] (ast.south) .. controls +(0,-0.5) and +(0,1) .. 
        node[operationleft, font=\footnotesize, align=center, yshift=-5mm, xshift=8mm] {Interpreter\\[-1mm]{\tiny \textcolor{gray}{interp.ml}}} (output.north);
    \draw[interp] (dast.south) .. controls +(0,-0.5) and +(0,1) .. (output.north);
\end{tikzpicture}
\caption{Architecture of the \Slice{} system. The \Slice{} compiler transforms continuous probabilistic programs into discrete ones through type-directed discretization. The resulting discrete program is then processed by the Dice inference engine to compute exact probability distributions. Dashed blue arrows show the interpreter's direct evaluation paths from AST and discretized AST to probability distributions.}
\label{fig:architecture}
\end{figure}