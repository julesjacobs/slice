% Listings setup
\lstset{
  basicstyle=\scriptsize\ttfamily,
  breaklines=true,
  columns=fullflexible,
  frame=none,
  backgroundcolor=\color{gray!5},
  showstringspaces=false
}

% Saveboxes for code blocks
\newsavebox{\boxone}
\begin{lrbox}{\boxone}
\begin{minipage}{5.2cm}
\begin{lstlisting}
let g = gaussian(0, 1) in
let check = g < 0.1 in
let x = if check then
    uniform(0, 0.2)
  else
    uniform(0.8, 1.0) in
x < 0.15
\end{lstlisting}
\end{minipage}
\end{lrbox}

\newsavebox{\boxtwo}
\begin{lrbox}{\boxtwo}
\begin{minipage}{5.2cm}
\begin{lstlisting}
let g = discrete(0.539828: 0#2, 0.460172: 1#2) in
let check = g <#2 1#2 in
  let x = if check then
        discrete(0.75: 0#2, 0.25: 1#2)
      else
        discrete(0: 0#2, 1: 1#2) in
    x <#2 1#2
\end{lstlisting}
\end{minipage}
\end{lrbox}

\newsavebox{\boxthree}
\begin{lrbox}{\boxthree}
\begin{minipage}{5.2cm}
\begin{lstlisting}
let g = discrete(0.539828, 0.460172) in
let check = g < int(1,1) in
  let x = if check then
        discrete(0.75, 0.25)
      else
        discrete(0., 1.) in
    x < int(1,1)
\end{lstlisting}
\end{minipage}
\end{lrbox}

\newsavebox{\boxfour}
\begin{lrbox}{\boxfour}
\begin{minipage}{5.2cm}
\begin{lstlisting}
(define g (if (flip 0.539828) 0 1))
(define check (< g 1))
(define x (if check 
               (if (flip 0.75) 0 1) 
               (if (flip 0.0) 0 1)))
(< x 1)
\end{lstlisting}
\end{minipage}
\end{lrbox}


\begin{center}
\begin{tikzpicture}[
    box/.style={
    draw,
    thick,
    align=left,
    inner sep=4pt,
    minimum width=5.4cm,
    minimum height=2.5cm
    },
    arrow/.style={
    -{Latex[length=2.5mm]},
    thick
    }
]

    % Top row
    \node[box] (box1) {\usebox{\boxone}};
    \node[box, right=1.6cm of box1] (box2) {\usebox{\boxtwo}};

    % Bottom row: tightly packed under box2
    \coordinate (belowcenter) at ($(box2.south) + (0,-2.2cm)$);
    \node[box, anchor=north east] (box3) at ($(belowcenter) + (-2.0cm, 0)$) {\usebox{\boxthree}};
    \node[box, anchor=north west] (box4) at ($(belowcenter) + (-1.0cm, 0)$) {\usebox{\boxfour}};

    % Arrows
    \draw[arrow] (box1) -- (box2);
    \draw[arrow] (box2.south) -- (box3.north);
    \draw[arrow] (box2.south) -- (box4.north);

\end{tikzpicture}
\end{center}